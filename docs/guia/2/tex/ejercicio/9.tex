\indent\underline{\textbf{Ejercicio 9}}\\
Demostrar que, dada una política estocástica $\pi (a \mid s)$, la función de valor de estado puede escribirse como

\[
    q_{\pi}(s,a) = \sum_{s',r} p(s', r \mid s, a) [r + \gamma v_{\pi}(s')]
\]

\indent\underline{\textbf{Solución}}\\
Sea,\\
$\pi(a \mid s)$: Política estocástica, es una función que asigna a cada estado $s$ una distribución de probabilidad sobre el conjunto de acciones $a$, es decir, $\pi(a \mid s) = P[A_t = a | S_t = s]$.\\
$q_{\pi}(s, a)$: Función de valor de acción bajo una política $\pi$, es el valor esperado del retorno a partir del estado $s$ y la acción $a$.\\
$p(s', r \mid s, a)$: Función de transición conjunta de estado y recompensas, es la probabilidad de que se obtenga el estado $s'$ y la recompensa $r$ al tomar la acción $a$ en el estado $s$.\\
$v_{\pi}(s)$: Valor de estado bajo la política $\pi$, es el valor esperado del retorno a partir del estado $s$.

La función de valor esperado de acción $q_{\pi}(s, a)$, se puede expresar como,
\[
    q_{\pi} (s,a) = E_{\pi}[R_t | S_t = s, A_t = a]
\]

Es decir, $q_{\pi}(s, a)$ es la suma de la recompensa inmediata $r$ y el valor esperado del retorno a partir del estado $s'$.
Al tomar la acción $a$ en el estado $s$ se obtiene la recompensa inmediata $r$ y se transiciona al estado $s'$.
Si se tiene en cuenta las recompensas futuras y todas las posibles transiciones y recompensas, la función de valor de acción se puede expresar como,

\[
    q_{\pi}(s,a) = \sum_{s',r} p(s', r \mid s, a) E[R_t \mid S_t = s', A_t]
\]

Las recompensas futuras se puede dividir en dos partes, que incluyen la recompensa inmediata $r$ y el valor esperado del retorno a partir del estado $s'$, ponderado por el factor de descuento $\gamma$,

\[
    E[R_t \mid S_t = s', A_t] = r + \gamma v_{\pi}(s')
\]

Al remplazar la expresión anterior en la función de valor de acción, se obtiene,

\[
    q_{\pi}(s,a) = \sum_{s',r} p(s', r \mid s, a) [r + \gamma v_{\pi}(s')]
\]

Entonces, el valor esperado total es igual al valor esperado de las recompensas inmediatas y futuras ponderadas por el factor de descuento $\gamma$~\cite{Sutton2018}.

\line(1,0){\textwidth}
