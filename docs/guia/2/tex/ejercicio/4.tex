\indent\underline{\textbf{Ejercicio 4}}\\
Para el proceso de Recompensas Markoviano (MRP) de la figura~\ref{fig:grafo_2}, el retorno $G_0$ de cada uno de los siguientes episodios con estado inicial C1 y $\gamma = 0.5$:

\begin{enumerate}
    \item C1 C2 C3 Pass Sleep
    \item C1 FB FB C1 C2 Sleep
    \item C1 C2 C3 Pub C2 C3 Pass Sleep
    \item C1 FB FB C1 C2 C3 Pub C1 FB FB FB C1 C2 C3 Pub C2 Sleep
\end{enumerate}

\begin{figure}[H]
    \centering
    \begin{tikzpicture}[->, >=stealth', auto, semithick, node distance=2cm]
      % Nodes
      \node[circle, draw] (facebook) {Facebook};
      \node[circle, draw, below=of facebook] (class1) {Class 1};
      \node[circle, draw, right=of class1] (class2) {Class 2};
      \node[circle, draw, right=of class2] (class3) {Class 3};
      \node[circle, draw, right=of class3] (pass) {Pass};
      \node[circle, draw, below=of class2] (pub) {Pub};
      \node[rectangle, draw, above=of class2] (sleep) {Sleep};

      % Arrows
      \path (facebook) edge[loop] node{0.9} (facebook)
            (facebook) edge[bend right] node[xshift=-20]{0.1} (class1)
            (class1) edge node[align=center, yshift=15, xshift=45]{\textcolor{red}{\(R=-1\)} \\ 0.5} (facebook)
            (class1) edge node[align=center, yshift=-15, xshift=-10]{0.5 \\ \textcolor{red}{\(R=-2\)}} (class2)
            (class2) edge node[align=left,  yshift=-15, xshift=-10]{0.8 \\ \textcolor{red}{\(R=-2\)}} (class3)
            (class2) edge node[align=center, xshift=40, yshift=15]{\textcolor{red}{\(R=0\)} \\ 0.2} (sleep)
            (pub) edge node[below, align=center,  yshift=-15, xshift=0] {0.4 \\ \\ \\ \textcolor{red}{\(R=+1\)}} (class2)
            (class3) edge node[align=center,  yshift=-15, xshift=-10]{0.6 \\ \textcolor{red}{\(R=-2\)}} (pass)
            (class3) edge node[right] {0.4} (pub)
            (pub) edge[bend left] node[above] {0.2} (class1)
            (pub) edge[bend right] node[left] {0.4} (class3)
            (pass) edge[bend right] node[align=center,  yshift=-45, xshift=120] {1.0 \\ \\ \\ \textcolor{red}{\(R=+10\)}} (sleep);
    \end{tikzpicture}
    \caption{Grafo de transición de estados}\label{fig:grafo_2}
\end{figure}

\indent\underline{\textbf{Solución}}\\
Sea,\\
$\gamma \in [0,1]$: Factor de descuento. $\gamma = 0.5$
$R_t \in \mathbb{R}$: Recompensa.\\
$G_t$: Retorno, es la acumulación de recompensas desde el instante $t+1$.

El retorno se calcula como la suma de las recompensas futuras descontadas a partir del tiempo $t$, es decir,

\[
    G_t = R_{t+1} + \gamma R_{t+2} + \gamma^2 R_{t+3} + \ldots
\]


\indent \textbf{Episodio 1:} C1 C2 C3 Pass Sleep\\
Se calcula el retorno,

\begin{align*}
    G_0 &= -2 + 0.5(-2) + 0.5^2(10) + 0.5^3(0) \\
    &= -2 - 1 + 2.5 + 0 \\
    &= -0.5
\end{align*}

\indent \textbf{Episodio 2:} C1 FB FB C1 C2 Sleep\\
Se calcula el retorno,

\begin{align*}
    G_0 &= -1 + 0.5(-1) + 0.5^2(-2) + 0.5^3(-2) + 0.5^4(0) \\
    &= -1 - 0.5 - 0.5 - 0.25 + 0 \\
    &= -2.25
\end{align*}

\indent \textbf{Episodio 3:} C1 C2 C3 Pub C2 C3 Pass Sleep\\
Se calcula el retorno,

\begin{align*}
    G_0 &= -2 + 0.5(-2) + 0.5^2(1) + 0.5^3(-2) + 0.5^4(-2) + 0.5^5(10) + 0.5^7(0) \\
    &= -2 - 1 + 0.25 - 0.25 - 0.125 + 0.3125 + 0 \\
    &= -2.8125
\end{align*}

\indent \textbf{Episodio 4:} C1 FB FB C1 C2 C3 Pub C1 FB FB FB C1 C2 C3 Pub C2 Sleep\\
Se calcula el retorno,

\begin{align*}
    G_0 &= -1 + 0.5(-1) + 0.5^2(-2) + 0.5^3(-2) + 0.5^4(-2) + 0.5^5(1) + 0.5^6(-2) + 0.5^7(-1) + \\
    & \quad 0.5^8(-1) + 0.5^9(-1) + 0.5^{10}(-2)+ 0.5^{11}(-2) + 0.5^{12}(-2) + 0.5^{13}(1) + 0.5^{14}(-2) + 0.5^{15}(0) \\
    &= -1 - 0.5 - 0.5 - 0.25 - 0.125 + 0.03125 - 0.03125 - 0.0078125 -0.0039063 - 0.0019531 - \\
    & \quad 0.0019531 - 0.0009766 - 0.0004883	+ 0.0001221 - 0.0001221 + 0 \\
    &= - 2.3921
\end{align*}

\line(1,0){\textwidth}
