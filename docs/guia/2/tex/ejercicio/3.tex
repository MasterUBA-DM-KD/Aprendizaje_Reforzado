\indent\underline{\textbf{Ejercicio 3}}\\
Demostrar que si la recompensa es constante $R_t = R \ \forall t$ y el factor de descuento $\gamma < 1$, entonces,

\[
    G_t = \frac{R}{1 - \gamma}
\]

\indent\underline{\textbf{Solución}}\\
El retorno $G_t$ es la suma de las recompensas futuras descontadas a partir del tiempo $t$, es decir,

\[
    G_t = R_t + \gamma R_{t+1} + \gamma^2 R_{t+2} + \ldots
\]

Dado que la recompensa es constante, $R_t = R \ \forall t$, entonces, $R_{t+1}=R$, $R_{t+2}=R$, $\ldots$, por lo que,

\[
    G_t = R + \gamma R + \gamma^2 R + \gamma^3 R + \ldots
\]

Tras factorizar toma la forma,

\[
    G_t = R(1 + \gamma + \gamma^2 + \ldots)
\]

La serie geométrica $1 + \gamma + \gamma^2 + \ldots$  es infinita. Sin embargo, dado que $\gamma < 1$, la serie converge a,

\[
    \sum_{k=0}^{\infty} \gamma^k = \frac{1}{1 - \gamma}
\]

Por lo tanto,

\[
    1 + \gamma + \gamma^2 + \ldots = \frac{1}{1 - \gamma}
\]

Que al remplazar en la expresión de $G_t$ se obtiene,

\begin{align*}
    G_t &= R \left(  \frac{1}{1 - \gamma} \right) \\
    G_t &= \frac{R}{1 - \gamma}
\end{align*}

\line(1,0){\textwidth}
