\indent\underline{\textbf{Ejercicio 2}}\\

Para el proceso de Markov, de la figura~\ref{fig:grafo_1}, calcular la probabilidad de cada uno de los siguientes episodios condicionados al estado inicial \textit{C1}:

\begin{enumerate}
    \item C1 C2 C3 Pass Sleep
    \item C1 FB FB C1 C2 Sleep
    \item C1 C2 C3 Pub C2 C3 Pass Sleep
    \item C1 FB FB C1 C2 C3 Pub C1 FB FB FB C1 C2 C3 Pub C2 Sleep
\end{enumerate}

\begin{figure}[H]
    \centering
    \begin{tikzpicture}[->, >=stealth', auto, semithick, node distance=2cm]
      % Nodes
      \node[circle, draw] (facebook) {Facebook};
      \node[circle, draw, below=of facebook] (class1) {Class 1};
      \node[circle, draw, right=of class1] (class2) {Class 2};
      \node[circle, draw, right=of class2] (class3) {Class 3};
      \node[circle, draw, right=of class3] (pass) {Pass};
      \node[circle, draw, below=of class2] (pub) {Pub};
      \node[rectangle, draw, above=of class2] (sleep) {Sleep};

      % Arrows
      \path (facebook) edge[loop right] node{0.9} (facebook)
            (facebook) edge[bend right] node{0.1} (class1)
            (class1) edge[bend right] node{0.5} (facebook)
            (class1) edge node{0.5} (class2)
            (class2) edge node{0.8} (class3)
            (class2) edge node{0.2} (sleep)
            (pub) edge node[below] {0.4} (class2)
            (class3) edge node{0.6} (pass)
            (class3) edge node[right] {0.4} (pub)
            (pub) edge[bend left] node[above] {0.2} (class1)
            (pub) edge[bend right] node[left] {0.4} (class3)
            (pass) edge[bend right] node[right] {1.0} (sleep);
    \end{tikzpicture}
    \caption{Grafo de transición de estados}\label{fig:grafo_1}
\end{figure}


\indent\underline{\textbf{Solución}}\\

\line(1,0){\textwidth}
