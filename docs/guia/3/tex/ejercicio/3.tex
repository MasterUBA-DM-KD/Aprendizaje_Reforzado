\indent\underline{\textbf{Ejercicio 3}}\\
En el Ejemplo 4.1 (\textit{GridWorld, Sutton\&Barto, 2018})~\cite{Sutton2018}, suponga que se agrega un nuevo estado $15$ del estado $13$ y sus acciones: \textit{left, up, right} y \textit{down}, lleva al agente a los estados $12$, $13$, $14$ y $15$, respectivamente.

\begin{itemize}
    \item Considere que las transiciones desde los estados originales no se cambian.
    ¿Cuánto vale $v_{\pi}(15)$ para la política $\pi$ aleatoria y equiprobable?
    Utilice $v(12) = -22$, $v(13) = -20$, $v(14) = -14$ (Fig. 4.1 del libro)
\end{itemize}

Justifique su respuesta.

\indent\underline{\textbf{Solución}}\\

\line(1,0){\textwidth}
