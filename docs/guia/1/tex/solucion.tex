\documentclass[12pt]{article}

\usepackage[margin=1in]{geometry}        % For setting margins
\usepackage{amsmath}                % For Math
\usepackage{fancyhdr}                % For fancy header/footer
\usepackage{graphicx}                % For including figure/image
\usepackage{cancel}                    % To use the slash to cancel out stuff in work

%%%%%%%%%%%%%%%%%%%%%%
% Set up fancy header/footer
\pagestyle{fancy}
\fancyhead[LO,L]{Alejandro Uribe}
\fancyhead[CO,C]{Aprendizaje Reforzado - Guía 1}
\fancyhead[RO,R]{\today}
\fancyfoot[LO,L]{}
\fancyfoot[CO,C]{\thepage}
\fancyfoot[RO,R]{}
\renewcommand{\headrulewidth}{0.4pt}
\renewcommand{\footrulewidth}{0.4pt}
%%%%%%%%%%%%%%%%%%%%%%

\begin{document}
    \underline{Ejercicio 1}

    \noindent Demostrar que,si conociéramos exactamente el valor de cada acción, es decir si  $Q_{t} (a) = E \left[ R_{t} \big| A_{t}=a \right] $, entonces la acción \textit{greedy} $ A_{t} = argmax_{a}Q_{t}(a) $ es la acción óptima en el sentido de que permite maximizar las recompensas totales

	\underline{Ejercicio 2}

    \noindent Lorem Ipsum

\end{document}
